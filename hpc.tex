\documentclass{article}
  \usepackage{graphicx}
  \usepackage{subcaption}
  \usepackage{listings}
  \usepackage{amsmath}
  \usepackage{caption}
  \usepackage{minted}
  
  \begin{document}
    \begin{titlepage}
        \vfill
        \begin{center}

          {\huge Part III Hybrid Photonics Computing \par}
          
          \
          
          \Large Lectured by Prof Natalia Berloff 
          
          \large Notes taken by Li Xi

          \ 

          Lent 2020 (16 Lectures)

          \

          \normalsize
          These notes are not endorsed nor checked by the lecturer and all errors are entirely mine. If you spot any error, raise an issue in the GitHub repository or create a pull request. 
          
        \end{center}
        
        \vfill

        \begin{flushleft}
          \Large \textbf{Summary}
        \end{flushleft}
        
        \noindent Recently Coherent Networks emerged as a promising alternative to universal classical or quantum computing and to quantum simulators/annealers. Their physical implementation is based on coherent dynamics of so-called coherent centers in the network of lasers, optical parametric oscillators, cold atomic gases, exciton-polariton condensates, memristors, VO2 oscillators, ring oscillators, multicore fibers, etc. They are expected to serve as fast and accurate accelerators for modern digital computers in the specialized tasks for NP-hard integer and continuous optimization problems in vastly different areas such as vehicle routing and scheduling problems, dynamic analysis of neural networks and financial markets, prediction of new chemical mate- rials and machine learning. The theoretical framework of coherent networks was proposed as heuristic algorithms for NP-hard optimization problems and efficient simulators for many-body systems. This course covers fundamental principles, algorithms, and applications, as well as the physical implementation of coherent network computing.

        \

        \begin{flushleft}
          \Large \textbf{Prerequisites}
        \end{flushleft}

        \noindent Undergraduate level degree in physics or applied mathematics and basic knowledge of scientific computing is expected.

        \ 

        \begin{flushleft}
          \Large \textbf{References}
        \end{flushleft}

        \begin{enumerate}
          \item P. R. Prucnal, B. J. Shastri \textit{Neuromorphic Photonics.} 1st edition. CRC Press, 2017.
          \item M. Newman \textit{}{Networks}. 2nd edition, Oxford University Press, 2018.
          \item K. Staliunas and VJ Sanches-Morcillo \textit{Transverse Patterns in Nonlinear Optical Resonators}, Springer, 2003.

        \end{enumerate}
        
        \vfill
    \end{titlepage}
  


  \setcounter{section}{-1}
  \section{Course outline}

  We will explore the following topics, though not necessarily in this order: 

  \begin{enumerate}
    \item Introduction to hybrid photonics computing as an alternative to classical von Neumann architecture and quantum computing. We will explore topics such as combinatorial optimisation and complexity theory (we will focus on P, NP, NP-complete and NP-hard classes). 
    \item Networks of coupled oscillators: Hopfield networks, Koramoto oscillators and other complex networks. 
    \item Spin Hamiltonians: Ising model, XY model, Potts Model, spin glasses, phase transitions and collective behaviour. 
    \item Physical systems for analogue computing: Laser arrays, Bose Einstein Condensates (both equilibrium and non-equilibrium condensates), optical parametric oscillators (also known as coherent Ising machines). 
    \item (If time permits) Advanced Topics: recurrent neural networks, machine learning, Boltzmann sampling algorithms, etc. 
  \end{enumerate}

  \noindent The examination at the end of the course will consist of three questions on the following topics: 

  \begin{itemize}
    \item Conversion of a problem into spin Hamiltonian 
    \item Network analysis 
    \item Physics and behaviour of systems
  \end{itemize}
  
  \ 

  \section{Introduction}
  
  Lorem ipsum dolor sit amet, consectetur adipiscing elit, sed do eiusmod tempor incididunt ut labore et dolore magna aliqua. Ut enim ad minim veniam, quis nostrud exercitation ullamco laboris nisi ut aliquip ex ea commodo consequat. Duis aute irure dolor in reprehenderit in voluptate velit esse cillum dolore eu fugiat nulla pariatur. Excepteur sint occaecat cupidatat non proident, sunt in culpa qui officia deserunt mollit anim id est laborum. 

  \end{document}