\documentclass{article}
  \usepackage{graphicx}
  \usepackage{subcaption}
  \usepackage{listings}
  \usepackage{amsmath}
  \usepackage{caption}
  \usepackage{minted}
  
  \begin{document}
    \begin{titlepage}
        \vfill
        \begin{center}

          {\huge Part III Hybrid Photonics Computing \par}
          
          \
          
          \Large Lectured by Prof Natalia Berloff 
          
          \large Notes taken by Li Xi

          \ 

          Lent 2020

          \

          \normalsize
          These notes are not endorsed nor checked by the lecturer and all errors are entirely mine. If you spot any error, raise an issue in the GitHub repository or create a pull request. 
          
        \end{center}
        
        \vfill

        \begin{flushleft}
          \Large \textbf{Summary}
        \end{flushleft}
        
        \noindent Recently Coherent Networks emerged as a promising alternative to universal classical or quan- tum computing and to quantum simulators/annealers. Their physical implementation is based on coherent dynamics of so-called coherent centers in the network of lasers, optical parametric oscillators, cold atomic gases, exciton-polariton condensates, memristors, VO2 oscillators, ring oscillators, multicore fibers, etc. They are expected to serve as fast and accurate accelerators for modern digital computers in the specialized tasks for NP-hard integer and continuous op- timization problems in vastly different areas such as vehicle routing and scheduling problems, dynamic analysis of neural networks and financial markets, prediction of new chemical mate- rials and machine learning. The theoretical framework of coherent networks was proposed as heuristic algorithms for NP-hard optimization problems and efficient simulators for many-body systems. This course covers fundamental principles, algorithms, and applications, as well as the physical implementation of coherent network computing.

        \

        \begin{flushleft}
          \Large \textbf{Prerequisites}
        \end{flushleft}

        \noindent Undergraduate level degree in physics or applied mathematics and basic knowledge of scientific computing is expected.

        \ 

        \begin{flushleft}
          \Large \textbf{References}
        \end{flushleft}

        \begin{enumerate}
          \item P. R. Prucnal, B. J. Shastri \textit{Neuromorphic Photonics.} 1st edition. CRC Press, 2017.
          \item M. Newman \textit{}{Networks}. 2nd edition, Oxford University Press, 2018.
          \item K. Staliunas and VJ Sanches-Morcillo \textit{Transverse Patterns in Nonlinear Optical Resonators}, Springer, 2003.

        \end{enumerate}
        
        \vfill
    \end{titlepage}
  
  \section{Introduction}
  
  Lorem ipsum dolor sit amet, consectetur adipiscing elit, sed do eiusmod tempor incididunt ut labore et dolore magna aliqua. Ut enim ad minim veniam, quis nostrud exercitation ullamco laboris nisi ut aliquip ex ea commodo consequat. Duis aute irure dolor in reprehenderit in voluptate velit esse cillum dolore eu fugiat nulla pariatur. Excepteur sint occaecat cupidatat non proident, sunt in culpa qui officia deserunt mollit anim id est laborum. 

%%%%%%%%%%%%%%%%%%%%%%%%%%%%%%%%%%%%%
  \newpage
  \clearpage
  \section{References}
  \begingroup
    \renewcommand{\section}[2]{}
    \bibliography{report}
    \bibliographystyle{apalike}
  \endgroup

  \end{document}